Collecting tweets from a geographically-defined set of users can be a challenging task due to the unreliable nature of Twitter's location-based search criteria. However, using a language filter based on a broad commonality like Spanish, then filtering based on country, seems to yield a higher volume of higher quality tweets, especially if one is interested in analyzing the views of local residents.

While the geoparsing process using BERT NER and Google Geocoding API shows promise, it is still unclear why it has not yielded better results in the given context. This method has its limitations, and more research is needed to improve its accuracy. Additionally, the small number of locations mentioned in the tweets compared to the total number of tweets and geotagged tweets indicates that they may not be of much use in practical applications.

It is essential to note that geoparsed and geotagged locations are not interchangeable, as the NLP methods used in geoparsing do not always accurately capture the intended location in a tweet. Therefore, it is important to use caution and context when interpreting geoparsed location data.

Overall, with careful selection of data collection methods and thorough data analysis, geotagged tweets can provide valuable insights into the attitudes and perspectives of people in different locations.