This study aims to investigate the potential of using Twitter data to track dengue virus transmission in South America, focusing on Argentina. Dengue is a viral disease transmitted primarily through mosquito bites and is endemic to several South American countries, including Argentina, where the virus has seen a rapid increase in transmission. Understanding travel patterns in South America is crucial for understanding local disease transmission patterns. The study identified two gaps in the literature: limited research on non-endemic countries and a need for tools designed explicitly for Argentina.

The study proposed three research questions to address these gaps:
\begin{enumerate}
    \item What methods can be used to gather tweets from a specific group of users based on their geographic location?
    \item Is it possible to use natural language processing (NLP) techniques to extract location information from tweets beyond geotagged information?
    \item Did geotagged locations and mentioned locations provide similar location measurements in tweets?
\end{enumerate}

The project's results indicated that the proposed data collection method was appropriate for monitoring regional human movement. However, the geoparsing technique employed had limited effectiveness, and it was observed that mentioned locations and geotags in tweets represent different information.